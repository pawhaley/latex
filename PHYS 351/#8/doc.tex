\documentclass[10pt,a4paper]{article}
\usepackage{amssymb}
\usepackage{fullpage}
\usepackage{graphicx}
\usepackage{amsmath}
\author{Parker Whaley}
\title{PHYS 351 \#8}
\begin{document}
\maketitle
\section{}
Consider the following processes by which one kilogram of liquid water is brought from an initial temperature of $0C$ to a final temperature of $100C$. Assume that the specific heat is $4186 \frac{J}{kgK}$ over the entire temperature range.
\subsection{}
The water is put in contact with a large heat reservoir at $100 ^\circ C$ . When the water has reached $100 ^\circ C$, what has been the change of entropy?
\subparagraph{}
Of the water?\\
We know that the relationship between changes in temperature and heat is $dq=cdT$ where $c=4186 \frac{J}{kgK} \cdot 1kg$.  We also know that $dS=\frac{dq}{T}$.  So $dS=\frac{cdT}{T}$ integrating we get $\delta S_w=c\ln(\frac{T_f}{T_i})=c\ln(\frac{T_H}{T_\circ})=1306\frac{J}{K}$.
\subparagraph{}
Of the heat reservoir?\\
Lets calculate the heat in $\delta q=\int cdT=c(T_H-T_\circ)=$.  So the hot reservoir gains $-\delta q$ heat at constant temperature thus $\delta S_H=-\delta q (\frac{1}{T_h})=c(\frac{T_\circ}{T_H}-1)=-1122\frac{J}{K}$
\subparagraph{}
Of the entire system (water plus heat reservoir)?\\
Of course this must be $\delta S_{tot}=\delta S_w+\delta S_H=c(\frac{T_\circ}{T_H}-1+\ln(\frac{T_H}{T_\circ}))$.  Plugging in $\delta S_{tot}=184\frac{J}{K}$
\subsection{}
The water is first put in contact with a large heat reservoir at $50^\circ C$. After the water has reached $50^\circ C$, it is put in contact with a large heat reservoir at $100^\circ C$. When the water has reached $100^\circ C$, what has been the change of entropy of the entire system (water plus both heat reservoirs)?\\
This is equivalent to doing the above procedure twice first with $T_\circ=0^\circ C$ and $T_H=50^\circ C$ then with $T_\circ=50^\circ C$ and $T_H=100^\circ C$.
$$c(\frac{T_0}{T_1}-1+\ln(\frac{T_1}{T_0})+\frac{T_1}{T_2}-1+\ln(\frac{T_2}{T_1}))=$$
$$c(\frac{T_0T_2+T_1T_1}{T_1T_2}+\ln(\frac{T_2}{T_0})-2)=97\frac{J}{K}$$
\subsection{}
Show how the water might be heated from $0^\circ C$ to $100^\circ C$ with no change in entropy of the entire system. In your discussion, comment on the necessity of using quasi-static steps to create reversible processes.\\
Suppose we considered taking this volume from a temperature T to another temperature $T+\delta$ by the process of exposing it to a reservoir at $T+\delta$.  We are only going to be interested in this process when $\delta$ is small we are then going to consider moving from T to $T_f$ a finite difference in temperature via infinitesimal steps.  The change in entropy over one of these steps would then be:
$$\delta S=c(\frac{T}{T+\delta}-1+\ln(\frac{T+\delta}{T}))=$$
$$c(\frac{1}{1+\delta/T}-1+\ln(1+\delta/T))$$
Considering $\delta /T$ is small we can expand the terms to get:
$$c([1-\delta/T]-1+[\delta/T])+O(\delta^2)=O(\delta^2)$$
Wow our $\delta S$ goes as $O(\delta^2)$ however our number of regions goes as $\delta$.  We can immediately say that in the limit as the difference in temperature goes to 0 that the sum of the entropy goes to 0 since we are summing up $\delta$ terms each of order $\delta^2$.  This process of putting the water in successively hotter and hotter reservoirs where the difference in the temperatures is vanishingly small has no change in entropy.
\section{}
The Clausius statement of the Second Law of Thermodynamics is:\\
No process is possible whose sole result is the transfer of heat from a colder body to a hotter body.\\
Show that the Clausius statement of the Second Law of Thermodynamics is consistent with the principle of the increase of entropy.\\
By our construction of entropy in class from the Clausius statement we may immediately say Clausius $\Rightarrow \delta S\geq 0$.  Let us now show that $\delta S\geq 0\Rightarrow$ Clausius.\\
We will proceed with a proof by contradiction technique.\\
Suppose there was a system that violates the Clausius statement but upholds the entropy statement.  Since it violates Clausius we can say that there must be two reservoirs $T_H$ and $T_C$ ($T_H>T_C$) where heat has moved from $T_C \rightarrow T_H$ and the rest of the system has remained unchanged.  Since the rest of the system is in it's original state we may say $\delta S_{other}=0$ since S is a state variable.  Let's denote the infinitesimal heat transferred as $\delta q$ note that $\delta S_H = \frac{\delta q}{T_H}$ and $\delta S_C = \frac{-\delta q}{T_C}$.  We now say that the total change in entropy is $\delta S=\delta S_{other}+\delta S_H+\delta S_C=q(\frac{1}{T_H}-\frac{1}{T_C})<0$.  Contradiction $\delta S \geq 0$ and $\delta S <0$ are incompatible thus our supposition is false and the negation of our supposition, $\delta S\geq 0\Rightarrow$ Clausius, is true.\\
We can now say $\delta S\geq 0\Leftrightarrow$ Clausius.
\end{document}