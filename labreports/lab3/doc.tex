\documentclass[12pt,a4paper]{article}
\usepackage{amssymb}
\usepackage{fullpage}
\usepackage{graphicx}
\author{Parker Whaley}
\title{lab \#3}
\begin{document}
\maketitle

\section{Tabulation of data}
Setting up the system to determine focal length we set up the rail as flat mirror convex lens needle magnifier and then positioned the lens and needle so that the needle was at the focal point.  All errors are the standard error for the track $\delta = .05cm$

\begin{tabular}{|l|l|l|}
\hline
lens surface & needle & lens \\
\hline
single lens convex side & 60.8cm & 43cm \\
\hline
single lens flat side & 61.9cm & 44cm \\
\hline
compound lens glued on side & 71.75cm & 56.25cm \\
\hline
compound lens other side & 64.75cm & 52.35cm \\
\hline

\end{tabular}\\

In this next section we found the nodal points for the compound lens. these measurements have error $\delta_2=.05mm$.

\begin{tabular}{|l|l|l|}
\hline
lens face & front of lens & nodal point\\
\hline
glued on face & 36.8mm & 13.7mm\\
\hline
other side & 48mm & 1.2mm\\
\hline

\end{tabular}

\section{Results}
First lets calculate the front and back focal lengths for our four lens surfaces.  Note that it will simply be the position of the needle minus the position of the lens as auto collimation occurs at the focal point.  Also note that the uncertainty in this subtraction will be $\delta\sqrt{2}=.071cm$.\\
\begin{tabular}{|l|l|}
\hline
lens surface & focus \\
\hline
single lens convex side & 17.8cm \\
\hline
single lens flat side & 17.9cm \\
\hline
compound lens glued on side & 15.5cm \\
\hline
compound lens other side & 12.4cm \\
\hline

\end{tabular}\\

Next we have the locations of the nodal points.  We can determine them in essentially the same way, the difference between the surface of the lens and the nodal points.  We then have a uncertainty of .071mm.\\
\begin{tabular}{|l|l|}
\hline
lens face & nodal point\\
\hline
glued on face & 23.1mm\\
\hline
other side & 46.8mm\\
\hline

\end{tabular}\\

\section{Additional discussion}
Are our results reasonable?\\
Yes, the two focal lengths for the single lens are about the same (theoretically we would expect these to be close but still not equal) and the two for the compound lens are both also small.  The nodes are both within the compound lens and I believe reasonable.\\

This method of determining the focal length is much more exact then the method used the previous week.  This is because there was a noticeable range of positions where the image would be in focus in the previous week and next to no range in this weeks measurements.\\

It is definitely OK for all but the most precise measurements to use the focal length of one side of a single lens for the other side.  However it would be unacceptable for the compound lens as the two focal lengths can very a lot since there is a large distance between the front of one lens and the front at the other side.\\

If we instead consider the distance between the focal point and nodal point on each side we get.

\begin{tabular}{|l|l|}
\hline
lens surface & focus-node distance \\
\hline
compound lens glued on side & 17.8cm \\
\hline
compound lens other side & 16.9cm \\
\hline

\end{tabular}\\

We see that there is still some difference between the two but that it is far reduced.  This error can probably be accounted for in the same light as the difference in the single lens,  these lenses have width and do not exactly satisfy the thin lens equation.















\end{document}