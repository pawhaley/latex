\documentclass[12pt,a4paper]{article}
\usepackage{amssymb}
\usepackage{amsmath}
\usepackage{fullpage}
\usepackage{graphicx}
\author{Parker Whaley}
\title{PHYS 462 (optics) HW\#6}
\begin{document}
\maketitle
\section{8.43}
This wave starts as $\tilde{E}=k\begin{bmatrix} 1 \\ i \\ \end{bmatrix}$.  The eighth wave plate will speed the x component by 1/8 its of a cycle so $\tilde{E}=k\begin{bmatrix} 1/\sqrt{2}(1-i) \\ i \\ \end{bmatrix}$.
\section{8.44}
Basically I see a polariser polarizing the light in one direction then the light passing through a half wave plate at $45^\circ$.  this causes the light to switch polarization so it is now rotated $90^\circ$ off of the original polariser.  Finally it passes through a polariser with the same polarization as the first polariser and is entirely blocked since it is at $90^\circ$.
\section{8.50}
From 8.32 $\Delta \phi=2\pi/\lambda l(\Delta n)=2\pi/\lambda l(\lambda K E^2)=2\pi l K (V/d)^2$
\section{8.54}
\subsection{a}
The first one is horizontal and the second is right handed of a intensity 3 times that of the first beam.
\subsection{b}
We have the Jones vectors $\begin{bmatrix} 1 \\ 0 \\ \end{bmatrix}$ and $\begin{bmatrix} \sqrt{3/2} \\ i\sqrt{3/2} \\ \end{bmatrix}$.  We can add these using POS to get $\begin{bmatrix} 1+\sqrt{3/2} \\ i\sqrt{3/2} \\ \end{bmatrix}$.  We then get a set of stokes parameters $(4+\sqrt{6},1+\sqrt{6},0,3+\sqrt{6})$.
\subsection{c}
The degree of polarization is one.
\subsection{d}
well this would be the polarizations $\begin{bmatrix} 1 \\ 0 \\ \end{bmatrix}$ and $\begin{bmatrix} 0 \\ 1 \\ \end{bmatrix}$ so we would get $\begin{bmatrix} 1 \\ 1 \\ \end{bmatrix}$. This is $(2,0,2,0)$.
\section{8.55}
$$\frac{1}{2} \begin{bmatrix} 1 & 0 & 1 & 0 \\ 0 & 0 & 0 & 0 \\ 1 & 0 & 1 & 0 \\ 0 & 0 & 0 & 0 \\ \end{bmatrix} \cdot \begin{bmatrix} 1 \\ 0 \\ 0 \\ 0 \\ \end{bmatrix}=\begin{bmatrix} 1/2 \\ 0 \\ 1/2 \\ 0 \\ \end{bmatrix}$$
This has half of the original radiance and is fully polarized.
\section{8.62}
In 60 we would see the same light emerging as we put in since all of the light is travelling along the slow axis.  For 61 we do have a component along the fast axis and another along the slow axis, since the beam is at $45^\circ$ to the polariser we know why will happen, it will change phase by $90^\circ$ and emerge as vertically polarized light.
\section{8.71}
These lenses will take vertically polarized light and make it circularly polarized.  The first one will take vertical light to R state and horizontal light to L state the second one does the opposite vertical to L horizontal to R.
\end{document}