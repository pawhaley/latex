\documentclass[12pt,a4paper]{article}
\usepackage{amssymb}
\usepackage{amsmath}
\usepackage{fullpage}
\usepackage{graphicx}
\author{Parker Whaley}
\title{PHYS 462 (optics) HW\#5}
\begin{document}
\maketitle
\section{6.8}
We will assume that the flower is short relative to the $4m$ distance to the lense, thus we can make the small angele approximation.  We will procede with a matrix approch.
$$
T_1=
\begin{bmatrix}
1 & 4\\
0 & 1
\end{bmatrix}$$

$$
R_2=
\begin{bmatrix}
1 & 0\\
5*(1/1.4-1) & 1/1.4
\end{bmatrix}$$

$$
T_3=
\begin{bmatrix}
1 & .2\\
0 & 1
\end{bmatrix}$$

$$
R_4=
\begin{bmatrix}
1 & 0\\
-2 & 1.4
\end{bmatrix}$$

$$
T_5=
\begin{bmatrix}
1 & L\\
0 & 1
\end{bmatrix}$$

$$
T_5 * R_4 * T_3 * R_2 * T_1=
\begin{bmatrix}
.714-3.429L&3-13L\\
-3.429 & -13
\end{bmatrix}$$

So we will have a image formed at $3-13L=0$ or $L=3/13m$ beyond the opposite side of the sphere.  At a magnification of $.714-3.429L=-0.077308$ so it will be inverted and smaller.

\section{6.9}
We can also do this with matraces
$$
R_1=
\begin{bmatrix}
1&   0\\
.029 &  2/3
\end{bmatrix}$$

$$
T_2=
\begin{bmatrix}
1 & 9\\
0 & 1
\end{bmatrix}$$

$$
R_3=
\begin{bmatrix}
1&   0\\
.075 &  1.5
\end{bmatrix}$$

$$
T_4=
\begin{bmatrix}
1 & L\\
0 & 1
\end{bmatrix}$$

$$
T_4*R_3*T_2*R_1=
\begin{bmatrix}
1.26+.138L&6+1.45L\\
.138&1.45
\end{bmatrix}$$

The point where $1.26+.138L=0$ will be our focus $f=-1.26/.138cm$ puting our fucus inside of the lense.

\section{6.14}
see fig1\\
We can use the thin lense equation to see what hapens to rais originating at infinity.  Passing thrugh the first lense they form a image at the focus.  In our thin lense equation for the second lense the new object would be at $S_o=-10cm$ so we get $1/-10+1/S_i=1/-20$ or $1/S_i=1/20$ or $S_i=20cm$ we end up with the focal point of the entre system at the right focal point of the diverging lense.

\section{6.16}
We will simply calculate the matrecies as before.
$$
R_1=
\begin{bmatrix}
1&   0\\
.01583 &  .79167
\end{bmatrix}$$

$$
T_2=
\begin{bmatrix}
1&   9.6\\
0 &  1
\end{bmatrix}$$

$$
R_3=
\begin{bmatrix}
1&   0\\
.01263 &  1.26316
\end{bmatrix}$$

$$
R_3*T_2*R_1=
\begin{bmatrix}
1.152&   7.6\\
0.034552&   1.096
\end{bmatrix}$$
using octave the det is 1.
\section{6.22}
$$
R_1=
\begin{bmatrix}
1.00000&   0.00000\\
-0.06667&   0.66667
\end{bmatrix}$$

$$
T_2=
\begin{bmatrix}
1&   1\\
0&   1
\end{bmatrix}$$

$$
R_3=
\begin{bmatrix}
1&   0\\
0 &  1.5
\end{bmatrix}$$

$$
R_3*T_2*R_1=
\begin{bmatrix}
0.93333&   0.66667\\
-0.10000&   1.00000
\end{bmatrix}$$
from octave the det is 1.
\section{6.24}
$$
R_1=
\begin{bmatrix}
1&   0\\
2/R&   1
\end{bmatrix}$$

$$
T_2=
\begin{bmatrix}
1&   d\\
0&   1
\end{bmatrix}$$



$$T_2*R_1*R_1*T_2*R_1$$

This will yeald the same matrix as the book (using octave), however it will be transposed since the book is working in the transposed space relative to us.  Also I note that the book is working in the right direction beeing positive, we take the direction the light is traveling to be positive.

Anyway if we plug in $d=r$ we get 
$$
M=
\begin{bmatrix}
-1&   0\\
0&-1
\end{bmatrix}$$
By inspection if we apply this matrix to itself we will get the identity matrix thus after four reflections we are left with the ray at its origional position and angle.
\section{6.28}
\subsection{a}
Spherical aberation since it is a rotationaly symetric aberation.
\subsection{b}
Coma.  See fig 6.23 it is identical.
\subsection{c}
I think this is astigmatism since it has a little of the cross patern, its a little bit of a bad image so I cant be sure.
\section{6.29}
\subsection{a}
Its got the cross pattern so astigmatism.
\subsection{b}
Coma.  See fig 6.23 in the book it looks very similar.


\end{document}