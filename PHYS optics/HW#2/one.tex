\documentclass[10pt,a4paper]{article}
\usepackage{amssymb}
\usepackage{fullpage}
\usepackage{graphicx}
\author{Parker Whaley}
\title{PHYS 462 (optics) HW\#2}
\begin{document}

\maketitle

\section{\#54}
What is the critical angle for total internal reflection in diamond?  Looking on table 4.1 we see that diamond has a reflective index $n_i=2.417$ we will assume outside of this diamond there is a vacuum or air or some other medium with a low index of reflection $n_t=1$.  We know total internal reflection occurs as the angle of the transmitted light with respect to the normal to the plane of incidence goes to $\pi/2$.  Using the law of transmission we get\[n_i \sin(\theta_i)=n_t \sin(\theta_t)\Rightarrow \arcsin[\frac{n_t}{n_i} sin(\theta_t)]=\theta_i=28^\circ\]This is a very shallow angle of total internal reflection, so we would expect light to get into the diamond and then bounce around a lot before escaping.  This would cause a well cut diamond to take a small bit of light and radiate it in all directions.

\section{\#57}
See fig1\\\\
We are asked to consider a fish looking upwards towards the surface.  Take the fish as looking at things in a cone of base angle $\theta$ the angle that the edge of that cone makes with the surface is then $\theta_i=\theta/2$.  We can then calculate the angle of transmission as $\sin(\theta_t)=n_i \sin(\theta_i)$ and the apparent cone (the base angle of the sky cone that gets compressed into the fish's cone as $\theta'=\theta_t*2$.  Now we need only ask what angle the fish's cone must be for the entire sky to be viable: \[\theta'=\pi\Rightarrow sin(\theta_t)=1=n_i\sin(\theta_i)\Rightarrow \theta=\theta_i*2=\arcsin(\frac{1}{n_i})*2=97^\circ\]
So the fish sees the entire sky shrunk from a disk of $180^\circ$ to a disk of $97^\circ$ because of the angles of transmission.

\section{\#65}
We are asked to compute the angle of total polarization of reflected light.  We can re-state this as the angle at witch $r_{\mid\mid}=0$ or using equation(4.38)
\[0=\frac{n_t}{\mu_t}\cos(\theta_i)-\frac{n_i}{\mu_i}\cos(\theta_t)\]
We also know that:
\[\sin(\theta_t)=\frac{n_i}{n_t}sin(\theta_i)\Rightarrow \cos(\theta_t)=\sqrt{1-\frac{n_i^2\sin^2(\theta_i)}{n_t^2}}\]
also recall:
\[\frac{c}{n}=\frac{1}{\sqrt{\mu\epsilon}}\Rightarrow (\frac{n_1}{n_2})^2=\frac{\mu_1\epsilon_1}{\mu_2\epsilon_2}\]
Now we can begin reducing:
\[0=\frac{n_t}{\mu_t}\cos(\theta_i)-\frac{n_i}{\mu_i}\cos(\theta_t)\]

\[\frac{n_t}{\mu_t}=\frac{n_i}{\mu_i\cos(\theta_p)}\sqrt{1-\frac{n_i^2\sin^2(\theta_p)}{n_t^2}}\]

\[(\frac{n_t\mu_i}{n_i\mu_t})^2=\frac{1}{\cos^2(\theta_p)}-\frac{n_i^2\tan^2(\theta_p)}{n_t^2}\]

\[(\frac{n_t\mu_i}{n_i\mu_t})^2=1+\tan^2(\theta_p)-\frac{n_i^2\tan^2(\theta_p)}{n_t^2}\]

\[\tan^2(\theta_p)=\frac{\frac{\epsilon_t\mu_i}{\epsilon_i\mu_t}-1}{1-\frac{\mu_i\epsilon_i}{\mu_t\epsilon_t}}\]

\[\tan(\theta_p)=\sqrt{\frac{\epsilon_t(\epsilon_t\mu_i-\epsilon_i\mu_t)}{\epsilon_i(\mu_t\epsilon_t-\mu_i\epsilon_i)}}\]

\section{\#70}
Calculate the transmittance of normal and parallel polarized waves:\\
(4.62-rewriting)
\[T=\frac{\sin(\theta_i)\cos(\theta_t)}{\sin(\theta_t)\cos(\theta_i)}t^2\]
(4.44)
\[t_\perp=\frac{2\sin(\theta_t)\cos(\theta_i)}{\sin(\theta_i+\theta_t)}\]
so:
\[T_\perp=\frac{\sin(\theta_i)\cos(\theta_t)}{\sin(\theta_t)\cos(\theta_i)}\frac{4\sin^2(\theta_t)\cos^2(\theta_i)}{\sin^2(\theta_i+\theta_t)}\]

\[T_\perp=\frac{4sin(\theta_i)\cos(\theta_t)\sin(\theta_t)\cos(\theta_i)}{\sin^2(\theta_i+\theta_t)}\]

knowing:

\[2\sin(\theta)\cos(\theta)=sin(2\theta)\]

we get

\[T_\perp=\frac{sin(2\theta_i)\sin(2\theta_t)}{\sin^2(\theta_i+\theta_t)}\]

now compute $T_\parallel$
\[t_\parallel=\frac{2\sin(\theta_t)\cos(\theta_i)}{\sin(\theta_i+\theta_t)\cos(\theta_i-\theta_t)}\]

\[T_\parallel=\frac{\sin(\theta_i)\cos(\theta_t)}{\sin(\theta_t)\cos(\theta_i)}\frac{4\sin^2(\theta_t)\cos^2(\theta_i)}{\sin^2(\theta_i+\theta_t)\cos^2(\theta_i-\theta_t)}\]

\[T_\parallel=\frac{4\sin(\theta_i)\cos(\theta_t)\sin(\theta_t)\cos(\theta_i)}{\sin^2(\theta_i+\theta_t)\cos^2(\theta_i-\theta_t)}\]

\[T_\parallel=\frac{sin(2\theta_i)\sin(2\theta_t)}{\sin^2(\theta_i+\theta_t)\cos^2(\theta_i-\theta_t)}\]

\section{\#71}
Show $R_\parallel+T_\parallel=1$ and $R_\perp+T_\perp=1$.

\[R_\parallel+T_\parallel=\frac{\tan^2(\theta_i-\theta_t)}{\tan^2(\theta_i+\theta_t)}+\frac{sin(2\theta_i)\sin(2\theta_t)}{\sin^2(\theta_i+\theta_t)\cos^2(\theta_i-\theta_t)}\]


\[\theta_i-\theta_t=\alpha\]
\[\theta_i+\theta_t=\beta\]
\[sin(2\theta_i)\sin(2\theta_t)=4\sin(\theta_i)\cos(\theta_t)\sin(\theta_t)\cos(\theta_i)=(2\cos(\theta_i)\cos(\theta_t))*(2\sin(\theta_t)\sin(\theta_i))=\]\[(cos(\alpha)+cos(\beta))(cos(\alpha)-cos(\beta))=cos^2(\alpha)-cos^2(\beta)\]
\[T_\parallel=\frac{cos^2(\alpha)-cos^2(\beta)}{\sin^2(\beta)\cos^2(\alpha)}\]

\[cos^2(\alpha)-cos^2(\beta)=cos^2(\alpha)-cos^2(\beta)+cos^2(\alpha)*cos^2(\beta)-cos^2(\alpha)*cos^2(\beta)=(1-cos^2(\beta))cos^2(\alpha)-(1-cos^2(\alpha))cos^2(\beta)=\]
\[sin^2(\beta)cos^2(\alpha)-sin^2(\alpha)cos^2(\beta)\]

By combining the above identity and our equation for $T_\parallel$:
\[T_\parallel=1-\frac{\tan^2(\alpha)}{\tan^2(\beta)}\]

Witch if we observe the above sum of $R_\parallel+T_\parallel$ we see that the tangent ratios cancel out and we are left with 1.\\\\

Now examine $R_\perp+T_\perp$:
\[R_\perp+T_\perp=\frac{\sin^2(\alpha)}{\sin^2(\beta)}+\frac{cos^2(\alpha)-cos^2(\beta)}{\sin^2(\beta)}\]

$sin(2\theta_i)\sin(2\theta_t)=cos^2(\alpha)-cos^2(\beta)$ was obtained in the previous section.

\[R_\perp+T_\perp=\frac{\sin^2(\alpha)+cos^2(\alpha)-cos^2(\beta)}{\sin^2(\beta)}=\frac{1-cos^2(\beta)}{\sin^2(\beta)}=\frac{sin^2(\beta)}{\sin^2(\beta)}=1\]

\section{\#78}
See fig2









\end{document}