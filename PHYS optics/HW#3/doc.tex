\documentclass[12pt,a4paper]{article}
\usepackage{amssymb}
\usepackage{fullpage}
\usepackage{graphicx}
\author{Parker Whaley}
\title{PHYS 462 (optics) HW\#3}
\begin{document}
\maketitle
\section{5.6}
Examining figure 5.6 I note that $\tan(\theta_i)=y_0/s_0$ and $\tan(\theta_t)=-y_i/s_i$ ($y_i$ is measured up from the axis).  Using the small angle approximation that $tan(\theta)=\sin(\theta)$ and Snell’s law we get $n_1y_0/s_0=-n_2y_i/s_i\Rightarrow M_T=y_i/y_0=-n_1s_i/n_2s_0$.

\section{5.7}
See fig1.  We know $n_1/s_0+n_2/s_i=\frac{n_2-n_1}{R}$ or (using $n_1=1$) $s_i=[n_2/R-n_2/s_i]^{-1}$ ($n_2=1.34$) so $s_i=4.48cm$.  Now we can use the resulted of the previous question to get the hight as .33cm.

\section{5.20}
Lensmaker's formula gives us $f=[(n_l-1)(1/R_1-1/R_2)]^-1=26.7cm$.  We can now calculate the image $d=[1/f-1/100]^-1=36cm$.  so the ant produces a real image to the left of the lens 36cm in front of the lens.

\section{5.22}
Using the lensemaker's formula we get a focal length of 100cm.  If this lens were put in water the focal length would increase, basically we see that the effect of the lens, that it bends the light via Snell’s law, is reduced if the reflective indexes of the two media become closer.

\section{5.25}
see fig2.  Distance between image and lens $s=[f^{-1}-s_0^{-1}]^{-1}=-7.5cm$ and has a hight of $6*7.5/10=4.5cm$.  So this lens will create a virtual image of the candle,right side up, slightly smaller then the real one and slightly closer to the lens.

\section{5.32}
see fig3.  Lets start by fixing the distance between the real and virtual images at L this means $L=s_0+s_1$ also note that we have our lens equation $1/s_0+1/s_1=1/f$.  Using the second equation to eliminate the $s_1$ dependence in the first equation we get $L=s_0+\frac{1}{1/f-1/s_0}=s_0+\frac{fs_0}{s_0-f}\Rightarrow 0=\frac{s_0^2-s_0L+fL}{s_0f} \Rightarrow 0=s_0^2-s_0L+fL \Rightarrow s_0=\frac{L\pm \sqrt{L^2-4fL}}{2}$ now we can talk about d the difference between these two locations where the image will be formed at our L $d=\frac{L+ \sqrt{L^2-4fL}}{2}-\frac{L- \sqrt{L^2-4fL}}{2}=\sqrt{L^2-4fL}$ we can then solve for $f=\frac{L^2-d^2}{4L}$.

\section{5.39}
see fig4.




























\end{document}