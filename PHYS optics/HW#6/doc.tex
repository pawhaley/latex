\documentclass[12pt,a4paper]{article}
\usepackage{amssymb}
\usepackage{amsmath}
\usepackage{fullpage}
\usepackage{graphicx}
\author{Parker Whaley}
\title{PHYS 462 (optics) HW\#6}
\begin{document}
\maketitle
\section{8.5}
$$\tilde{E}=\frac{\hat{i}-\hat{j}}{\sqrt{2}}E_\circ e^{(\omega t+(x+y)/\sqrt{2}\lambda+\pi/2)i}$$
\section{8.9}
We are told that the angle between the light and the polorizer is $60^\circ$ we also know that the intencity goes as $\cos^2(\theta)$ of the incident light.  So the intencity will be $(\cos(60))^2=(\frac{1}{2})^2=.25$ so $25\%$ of the light will be transmitted.
\section{8.14}
We will assume that the polorizer is ideal given no other information.  We then see that there is a $60^\circ$ angle between the polorizer and the incident light.  From the previous section we know that the intencity will be quartered to $50W/m^2$.
\section{8.18}
If we take non-polorized light and polorize it the intencity is reduced by the average value of $\cos^2(x)$ wich is $\frac{1}{2}$ so the first polorizer reduces the intencity by .5.  if we then have a polorizer at $50^\circ$ off of the privious one we will reduce the intecity by another $\cos^2(50)$ so our intencity will be $I=I_\circ\cdot.5cos^2(50^\circ)=207W/m^2$.  With the third polorizer in place we then get $I=I_\circ\cdot.5cos^2(25^\circ)cos^2(25^\circ)=337W/m^2$.
\section{8.23}
The light leaving the paper has lost its polorization since the paper absorbs the light and then transmits the light.
\section{8.30}
This will be the brewster's angle so (8.25) $\tan(\theta)=n=1.39$.
\section{8.35}
In the cristol the ordinary waves have a index of 1.5443 so there wevelength is $381.60nm$ and there frequency is preserved at $c/\lambda=5.09E14Hz$.  The extraordinary light has a index of 1.5534 and so will have a wevelength of $379.36nm$ and the same frequency of $5.09E14Hz$.
\end{document}