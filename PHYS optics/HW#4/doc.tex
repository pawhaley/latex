\documentclass[12pt,a4paper]{article}
\usepackage{amssymb}
\usepackage{fullpage}
\usepackage{graphicx}
\author{Parker Whaley}
\title{PHYS 462 (optics) HW\#4}
\begin{document}
\maketitle
\section{\#43}
In P.5.43a the light goes strait through $L_2$ without deviating.  We would expect in reality that the light would bend away from the optic axis and so the image would appear further away.\\

In P.5.43b we see beam 4 hitting $L_1$ parallel to the optic axis.  We would expect this beam to head towards $F_1'$, however in the provided illustration it goes through the center of $L_2$ instead.
\section{\#47}
see fig1
\section{\#54}
The first image is formed by the lens $\frac{1}{S_o}+\frac{1}{S_i}=\frac{1}{f}$ so $S_i=[\frac{1}{f}-\frac{1}{S_o}]^{-1}=[\frac{1}{50}-\frac{1}{250}]^{-1}=62.5cm$.  So the first image is $62.5cm$ behind the lens or $187.5cm$ in front of the mirror.  The plane mirror thus generates an image $187.5cm$ behind itself, this is the second image.  the lens then takes this image and generates a new image $S_i=[\frac{1}{f}-\frac{1}{S_o}]^{-1}=[\frac{1}{50}-\frac{1}{250+187.5}]^{-1}=56.5cm$ in front of the lens our third image.
\section{\#64}
We know that the girl's image in the plane mirror is the same as her actual hight, thus in the convex mirror she will appear as half of her hight so $S_i=-1/2S_o$.  Now we can easily calculate the focal length $\frac{1}{5}+\frac{-2}{5}=\frac{1}{f}$.  $f=-5$.
\section{\#65}
We can calculate the image made by the first mirror as $\frac{1}{S_o}+\frac{1}{S_i}=\frac{1}{f}$ where $S_o=\infty$.  So $S_i=f=1m$.  To the second mirror this object is located behind the mirror by $1/4m$ so for it $S_o=25cm$.  Now we can calculate where its image will be, $\frac{1}{-25}+\frac{1}{S_i}=\frac{1}{-30}$ so $S_i=150cm$, witch puts the image $75cm$ behind the main mirror.
\section{\#78}
Examining the image carefully I note that some of the rays originating from the top of the object end up on the bottom of the image and some end up on the top of the image, this is all wrong!
\section{\#85}
We would need a converging lens with a $S_o$ at $25cm$ becoming a virtual image at $S_i=-125$.  This gives us a focal point at $31.25cm$ or $3.2$ dioptre lens.


\end{document}