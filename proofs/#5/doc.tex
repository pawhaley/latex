\documentclass[12pt,a4paper]{article}
\usepackage{amssymb}
\usepackage{fullpage}
\usepackage{graphicx}
\author{Parker Whaley}
\title{Proof HW}
\begin{document}
\maketitle
\section{2.1.13}
Theorem.  For all $a\neq 0$,$(a^{-1})^{-1}=a$.\\
Proof.\\
Consider a non-zero number, lets call it a.  Note that by the definition of multiplicative inverse $a^{-1}\cdot a=1$.  Note $a^{-1}\cdot a=1\Rightarrow (a^{-1})^{-1}\cdot a^{-1}\cdot a=(a^{-1})^{-1}$.  Noting that $(a^{-1})^{-1}\cdot a^{-1}=1$ by the definition of multiplicative inverse we see that $(a^{-1})^{-1}\cdot a^{-1}\cdot a=(a^{-1})^{-1}\Rightarrow  a=(a^{-1})^{-1}$.  Since we have concluded that $a=(a^{-1})^{-1}$ for a arbitrary non-zero a we conclude that the theorem holds for all non-zero numbers.\\$\Box$
\section{2.1.15}
Note that $1\cdot 1=1$.  Also note that $1^{-1}\cdot 1=1$ by the definition of multiplicative inverse.  We conclude that one is a value for the multiplicative inverse of one, and as shown in class there is only one multiplicative inverse for a number thus $1^{-1}=1$.
\section{2.1.16}
Theorem.  For all $a\in \mathbb{R}$,$(-1)\cdot a=-a$.\\
Proof.\\
Select a arbitrary element of $\mathbb{R}$ lets call it a.  First note that by the definition of additive inverse $1+(-1)=0$.  Note $1+(-1)=0 \Rightarrow a(1+(-1))=a\cdot 0 \Rightarrow a+(-1)\cdot a=0 \Rightarrow (-a)+a+(-1)\cdot a=(-a)+0 \Rightarrow (-1)\cdot a=(-a)$.  Since we have shown that the premise holds for an arbitrary element of $\mathbb{R}$ it must hold for all elements of $\mathbb{R}$.\\$\Box$

\end{document}