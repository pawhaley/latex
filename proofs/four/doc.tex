\documentclass[12pt,a4paper]{article}
\usepackage{amssymb}
\usepackage{fullpage}
\usepackage{graphicx}
\author{Parker Whaley}
\title{Proof HW}
\begin{document}
\maketitle
\section{3.6.13}
We are asked to consider the set of all sets, F.  We then define the $\equiv$ operator to be, for sets A and B, $A\equiv B \Leftrightarrow A\subset B \vee B\subset A$.  Is $\equiv$ a equivalence relation on F?\\

No, I will demonstrate that it is not with a proof by contradiction.\\

Suppose that the operator $\equiv$ is a equivalence relation.\\
Consider the sets $A=\{ 1\}$,  $B=\{ 1,2\}$,  $C=\{ 2\}$.  Note that $A\equiv B$ since $A\subseteq B$.  Also note that $B\equiv C$ since $C\subseteq B$.  Since $\equiv$ is a equivalence relation amongst A, B, and C we can use the transitive property to say that $A\equiv B \wedge B\equiv C \rightarrow A\equiv C$.  Note that, since $A\nsubseteq C\wedge C\nsubseteq A\Rightarrow \neg(A\subseteq C\vee C\subseteq A)$, DeMorgan’s law, it must be that $A\not\equiv C$.\\
We have thus reached a contradiction, it is not possible for $A\equiv B$ and $A\not\equiv B$ so our initial supposition that $\equiv$ is a equivalence operator must be false.

\section{3.6.15}
We are asked to consider F, the set of all non empty sets.  We define $\equiv$ to be, for sets A and B, $A\equiv B \Leftrightarrow$ $A\cap B$ is a non-empty set.  Is $\equiv$ a equivalence relation on F?\\

No, I will demonstrate that it is not with a proof by contradiction.\\

Suppose that the operator $\equiv$ is a equivalence relation.\\
Consider the sets $A=\{ 1\}$,  $B=\{ 1,2\}$,  $C=\{ 2\}$.  Note that $A\equiv B$ since $A\cap B=\{ 1\}$.  Also note that $B\equiv C$ since $C\cap B=\{ 2\}$.  Since $\equiv$ is a equivalence relation amongst A, B, and C we can use the transitive property to say that $A\equiv B \wedge B\equiv C \rightarrow A\equiv C$.  Note that, since $A\cap C=\varnothing $ it must be that $A\not\equiv C$.\\
We have thus reached a contradiction, it is not possible for $A\equiv B$ and $A\not\equiv B$ so our initial supposition that $\equiv$ is a equivalence operator must be false.

\subsection{3.6.17}
Is $\neq$ an equivalence relation on $\mathbb{R}$?\\

No, I will demonstrate this with a proof by contradiction.\\

Suppose $\neq$ is a equivalence relation on $\mathbb{R}$.\\
Consider $a=0$.  Note that $a\in \mathbb{R}$.  Thus since $\neq$ is a equivalence relation and therefore must be reflexive we can say $a\neq a$.  However we know $0=0$ thus $a=a$.\\
We have reached a contradiction it is impossible for $a\neq a$ and $a=a$.  Thus our initial supposition that $\neq$ is a equivalence relation must be false.



\end{document}